\chapter{Results}
\label{ch:results}

\section{Scaling Study: Model Saturation}
The primary objective of this thesis was to determine the scalability of BMI prediction models. We trained the pipeline on increasing subset sizes. The results are summarized in Figure \ref{fig:saturation}.

\begin{figure}[H]
    \centering
    \includegraphics[width=0.9\textwidth]{kuvat/saturation_curve.png}
    \caption{\textbf{Model Performance vs. Sample Size.} The Root Mean Squared Error (RMSE) decreases as the number of samples increases from 1k to 10k. A significant improvement is observed between 1k and 5k, with the curve beginning to flatten (saturate) between 5k and 10k, indicating diminishing returns for additional data.}
    \label{fig:saturation}
\end{figure}

\section{Performance Analysis}
\begin{itemize}
    \item \textbf{1k Samples:} High variance and higher RMSE. The model struggles to generalize.
    \item \textbf{5k Samples:} Significant drop in RMSE. The variance between seeds (red error bars) also decreases, indicating a more stable model.
    \item \textbf{10k Samples:} The best performance, but the marginal gain over 5k is smaller than the gain from 1k to 5k.
    \item \textbf{13k Samples:} Achieve the lowest RMSE using 13,000 samples (5.70). The model effectively utilizes the additional data to refine predictions.
    \item \textbf{15k Samples:} Performance plateaus and slightly reverts (RMSE 5.78), confirming that adding more data beyond 13k does not yield significant improvements and may introduce noise.
\end{itemize}

This confirms that a dataset of 13,000 samples captures the maximal predictive signal for BMI in this cohort, with diminishing returns observed at 15,000 samples.

\section{Feature Importance Analysis}
To understand the biological drivers of the prediction, we extracted the top 20 most predictive features from the 10,000-sample model (Figure \ref{fig:featimp}). The importance scores represent the absolute magnitude of the GLM coefficients, identifying the microbial taxa that contribute most strongly to the BMI prediction.

\begin{figure}[H]
    \centering
    \includegraphics[width=1.0\textwidth]{kuvat/feature_importance_10k.png}
    \caption{\textbf{Top 20 Predictive Microbial Features.} Features are ranked by their contribution to the BMI prediction model. Identification of specific taxa aligns with known literature on obesity-associated microbiome signatures.}
    \label{fig:featimp}
\end{figure}
