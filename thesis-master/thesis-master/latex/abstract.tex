\begin{abstract}
The human gut microbiome is a complex ecosystem increasingly linked to various health outcomes, including metabolic disorders such as obesity. Machine learning (ML) offers a powerful approach to predict host traits like Body Mass Index (BMI) from high-dimensional microbial abundance data. However, the computational resources required for such analyses often exceed the capabilities of standard local hardware, creating a barrier to accessibility and reproducibility.

This thesis investigates the scalability and optimization of ML workflows for microbiome-based BMI prediction. We implemented a reproducible pipeline using the \texttt{mikropml} R package and the Snakemake workflow management system. A comprehensive "Scaling and Saturation Study" was conducted to evaluate model performance across dataset sizes ranging from 1,000 to 10,000 samples. We addressed the "curse of dimensionality" through rigorous feature prefiltering, removing rare species ($<1\%$ prevalence) to reduce memory overhead by approximately 80\% without compromising predictive accuracy.

Our results demonstrate that model performance (RMSE) improves significantly as sample size increases from 1,000 to 5,000, establishing a clear saturation curve. Furthermore, we show that optimized feature selection enables effective training on local hardware, democratizing access to microbiome ML tools. This work contributes to the best practices for robust, reproducible, and scalable microbiome data analysis, aligning with recent initiatives such as ML4Microbiome.
\end{abstract}
