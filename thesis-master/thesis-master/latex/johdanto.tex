\chapter{Introduction} \label{Introduction}

Obesity is a global health crisis, with the World Health Organization reporting that over 650 million adults were obese as of 2016. While diet and genetics are primary drivers, recent research has highlighted the critical role of the gut microbiome—the diverse community of microorganisms residing in the human gastrointestinal tract—in energy harvest and metabolic regulation.

\section{Microbiome and BMI}
Dysbiosis, or the imbalance of microbial communities, has been linked to metabolic phenotypes. Studies have shown that gut microbiota from obese individuals can induce weight gain in germ-free mice, suggesting a causal component. However, translating these biological insights into predictive markers for Body Mass Index (BMI) in humans remains a complex challenge due to the immense heterogeneity of the microbiome across individuals.

\section{Computational Challenges}
Analyzing microbiome data involves processing high-dimensional feature matrices where the number of predictors (bacterial species) often rivals or exceeds the number of samples ($p >> n$). Furthermore, metagenomic data is sparse (zero-inflated) and compositional. 
Standard machine learning approaches, such as Random Forest or Elastic Net regression, are powerful tools for this task but can be computationally expensive. Training non-linear models on cohorts with tens of thousands of samples requires significant Random Access Memory (RAM) and processing time.

\section{Thesis Objectives}
The primary objective of this thesis is to develop a reproducible machine learning pipeline to predict BMI from taxonomic profiles using the \texttt{mikropml} R package. Beyond simple prediction, this work specifically addresses the \textbf{scaling limitations} of such pipelines.
We aim to:
\begin{enumerate}
    \item Develop a Snakemake-based workflow for reproducible analysis.
    \item Evaluate the impact of feature filtering (dimensionality reduction) on memory usage and model accuracy.
    \item Perform a saturation analysis to determine the optimal sample size required for robust BMI prediction.
\end{enumerate}

By optimizing these parameters, we demonstrate that effective microbiome-based predictive modeling can be performed on standard hardware without relying exclusively on supercomputing resources.
